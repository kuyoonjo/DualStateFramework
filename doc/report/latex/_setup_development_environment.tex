The operating system of my laptop is Mac O\+S X Yosemite \cite{osxyosemite}. I have installed V\+Mware Fusion \cite{vmwarefusion} and have created two virtual machine for other operating systems(Microsoft Windows 7 \cite{microsoftwindows7} and Ubuntu 14.\+04 L\+T\+S \cite{ubuntu14_04LTS}). The development will be mainly taken place under Mac, and tested in all these three operating systems. \hypertarget{_setup_development_environment_SetupDevelopmentEnvironmentCppCompilerandIDE}{}\section{C++ Compiler and I\+D\+E}\label{_setup_development_environment_SetupDevelopmentEnvironmentCppCompilerandIDE}
\hypertarget{_setup_development_environment_SetupDevelopmentEnvironmentCppCompilerandIDEMacOSX}{}\subsection{Mac O\+S X}\label{_setup_development_environment_SetupDevelopmentEnvironmentCppCompilerandIDEMacOSX}
To setup Xcode \cite{xcode} and default C++ compiler L\+L\+V\+M \cite{llvm} in Mac O\+S X is very easy.
\begin{DoxyItemize}
\item Open App Store → search \char`\"{}\+Xcode\char`\"{} → click \char`\"{}get\char`\"{} button
\end{DoxyItemize}

After the installation, the I\+D\+E Xcode and the compiler L\+L\+V\+M is installed. To see the information about clang by command \char`\"{}clang -\/-\/version\char`\"{}. 
\begin{DoxyCode}
MacBook-Pro:~ yuchen$ clang --version
Apple LLVM version 6.1.0 (clang-602.0.49) (based on LLVM 3.6.0svn)
Target: x86\_64-apple-darwin14.3.0
\end{DoxyCode}
 \hypertarget{_setup_development_environment_SetupDevelopmentEnvironmentCppCompilerandIDEMicrosoftWindows}{}\subsection{Microsoft Windows}\label{_setup_development_environment_SetupDevelopmentEnvironmentCppCompilerandIDEMicrosoftWindows}
Download Visual Studio Community \cite{visualstudiocommunity} from \href{https://www.visualstudio.com}{\tt https\+://www.\+visualstudio.\+com} and install it. Visual Studio has its own embedded C++ compiler. \hypertarget{_setup_development_environment_SetupDevelopmentEnvironmentCppCompilerandIDEUbuntu}{}\subsection{Ubuntu}\label{_setup_development_environment_SetupDevelopmentEnvironmentCppCompilerandIDEUbuntu}
Open a terminal and type following command to install the G\+N\+U Compiler Collection(\+G\+C\+C) \cite{gcc} 
\begin{DoxyCode}
sudo apt-\textcolor{keyword}{get} install build-essential
\end{DoxyCode}
 Type \char`\"{}gcc -\/v\char`\"{} to print the descriptions. 
\begin{DoxyCode}
yu@ubuntu:~$ gcc -v
Using built-in specs.
COLLECT\_GCC=gcc
COLLECT\_LTO\_WRAPPER=/usr/lib/gcc/x86\_64-linux-gnu/4.8/lto-wrapper
Target: x86\_64-linux-gnu
Configured with: ../src/configure -v --with-pkgversion=\textcolor{stringliteral}{'Ubuntu 4.8.2-19ubuntu1'} --with-bugurl=file:
Thread model: posix
gcc version 4.8.2 (Ubuntu 4.8.2-19ubuntu1) 
\end{DoxyCode}
\hypertarget{_setup_development_environment_SetupDevelopmentEnvironmentThirdpartylibraries}{}\section{Third party libraries}\label{_setup_development_environment_SetupDevelopmentEnvironmentThirdpartylibraries}
\hypertarget{_setup_development_environment_SetupDevelopmentEnvironmentThirdpartylibrariesIntelThreadingBuildingBlocks}{}\subsection{Intel Threading Building Blocks}\label{_setup_development_environment_SetupDevelopmentEnvironmentThirdpartylibrariesIntelThreadingBuildingBlocks}
Intel T\+B\+B \cite{inteltbb} is a C++ library for parallel computing. Use Homebrew \cite{homebrew} to Install the library for Mac O\+S X\+: 
\begin{DoxyCode}
$ brew install libtbb
\end{DoxyCode}
 For Ubuntu use following command 
\begin{DoxyCode}
$ sudo apt-\textcolor{keyword}{get} install libtbb2
\end{DoxyCode}
 For Window need go to its download page \href{https://www.threadingbuildingblocks.org/download}{\tt https\+://www.\+threadingbuildingblocks.\+org/download}. Download the Windows version binaries package and unzip it.
\begin{DoxyItemize}
\item First, create a a new folder. The full path of it is \char`\"{}\+C\+:\textbackslash{}\+Program Files\textbackslash{}tbb\char`\"{}. Inside the folder create three folders \char`\"{}include\char`\"{}, \char`\"{}lib\char`\"{}, and \char`\"{}bin\char`\"{}.
\item Secondly, go to the unzipped folder. Copy everything inside the \char`\"{}include\char`\"{} to the \char`\"{}include\char`\"{} has just been created.
\item Inside the \char`\"{}lib\char`\"{}, there are two folders \char`\"{}ia32\char`\"{} and \char`\"{}intel64\char`\"{} that means the architecture of operating system. My Window 7 is 32 bits, so I chose the \char`\"{}ia32\char`\"{}. Again, inside \char`\"{}ia32\char`\"{}, there are many folders those are the versions of Visual Studio. The version of Visual Studio that I installed is 2013. The core of Visual Studio 2013 is vs2012, so copy all files inside \char`\"{}vc12\char`\"{} to the folder \char`\"{}lib\char`\"{} which has just been created.
\item Inside the \char`\"{}bin\char`\"{}, do the same thing as \char`\"{}lib\char`\"{} did.
\item Create a environment variable \char`\"{}tbb\char`\"{} refers to the directory \char`\"{}\+C\+:\textbackslash{}\+Program Files\textbackslash{}tbb\char`\"{}.
\end{DoxyItemize}\hypertarget{_setup_development_environment_SetupDevelopmentEnvironmentThirdpartylibrariesSFML}{}\subsection{S\+F\+M\+L}\label{_setup_development_environment_SetupDevelopmentEnvironmentThirdpartylibrariesSFML}
Simple and Fast Multimedia Library(S\+F\+M\+L \cite{sfml}) is a graphics library. For Mac, just go to the download page \href{http://www.sfml-dev.org/download/sfml/2.1/}{\tt http\+://www.\+sfml-\/dev.\+org/download/sfml/2.\+1/} to get the installer and run it. For Ubuntu, just need one line command\+: 
\begin{DoxyCode}
sudo apt-\textcolor{keyword}{get} install libsfml-dev
\end{DoxyCode}
 However, there is a problem for Windows. When I went to its download page, I could not find the version for Visual Studio 2013. The solution is to compile the source code which can be cloned from \href{https://github.com/LaurentGomila/SFML.git}{\tt https\+://github.\+com/\+Laurent\+Gomila/\+S\+F\+M\+L.\+git}.\hypertarget{_setup_development_environment_SetupDevelopmentEnvironmentOtherTools}{}\section{Other Tools}\label{_setup_development_environment_SetupDevelopmentEnvironmentOtherTools}

\begin{DoxyItemize}
\item Source\+Tree \cite{sourcetree}, a G\+U\+I git front-\/end, which is available in both Mac and Windows.
\item cmake-\/gui \cite{cmakegui}, the official G\+U\+I front-\/end for C\+Make, available in Mac, Windows, and Linux. 
\end{DoxyItemize}