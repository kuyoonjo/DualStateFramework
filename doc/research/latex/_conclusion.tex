After looking into all of the differences, I have decided to use Intel T\+B\+B to develop the library. Open\+M\+P is easy to use however, it uses compiler directives, which means not all compiler support Open\+M\+P. For example, the default compiler in Mac O\+S X does not support Open\+M\+P. Intel T\+B\+B is a library developed by C++ so that any C++ compiler can use it. ~\newline
~\newline
Because Intel T\+B\+B is a C++ library, so the programming language is C++. The graphic library will be S\+F\+M\+L because of the object oriented programming. Also, S\+F\+M\+L is available in multiple platform. The C++ I\+D\+E is Xcode, which has the best performance under Mac system. It has a serious of tools for debugging such as memory leak detector. Also, use Xcode to create Mac application bundles will be very simple. However, this project is not only for Mac users, so I will use C\+Make to make the project portable for other operating systems. ~\newline
~\newline
I will use Git for source control because it is the most powerful and most widely used. Github provides a free service that user can create public repositories for free. Doxygen, Graphviz, and La\+Tex will be used for documentation. Doxygen can generate document from code. Graphviz can create class diagrams from code. The reason I choose La\+Tex indeed of visible document editors like Microsoft Word, Google Doc, or Apple Pages is that Doxygen can only generate html and latex documents. ~\newline
~\newline
Package\+Maker, Installforge, and Debreate will be used for Mac O\+S X, Microsoft Window, and Debian-\/like Linux. All of them are G\+U\+I and simple. 